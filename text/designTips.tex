%!TEX root = ../abgabe.tex

\section{Design Tips}

Geschrieben von: Saulius A

In diesen Kapitel werde ich ihnen verschiedene Gestaltungsprinzipien vorstellen, die bei der Entwicklung von Mobilen Applikationen zu beachten sei. Der Hauptmerk dieser Prinzipien sind die Smartphones, aber es kann auch für Software auf andere Geräte angewendet werden, die in Mobilen Kontexten bedient werden müssen. Als weiteres werde ich noch Designprinzipien bei Wearable Computing vorstellen. 

\subsection{Gestaltungs Regeln für Smartphone Apps} % (fold)
\label{sub:design_f_r_mobile_ger_te}

Seid der Einführung des iPhones, durchdingen die Smarthphones den Markt mit eine unfasbare geschwindigkeit. Der Markt ist groß, und der Wille von den Benutzern ihren neuen Gerät auch unterwegs zu benutzen wird immer Größer. So stellt sich die Frage, welche Prinzipien sollte man beachten bei erstellung von Applikationen für unterwegs. Soll die vorhandenen Programme, die wir von stationären Rechnern kennen, einfach für bedienung mit Kleineren bildschirm angepasst, oder soll viel mehr als nur die Benutzeroberfläche verändert werden? Es gibt viele Antworten, die ich mit der Designregeln vorstellen werde.

\subparagraph{Entferne das Fett} % (fold)
\label{subp:entferne_das_fett}

Wie in der Einführung erwähnt wurde, hat der Benutzer im mobilen Kontext eine kurze Zeitspanne, in der er sich auf den mobilen Gerät konzentrieren kann. So kann zu viel Funktionalität den Benutzer mehr stöhren als helfen. So ist es ratsam die wichtigen funktionaliät anzubieten, sowie die Informationsdichte zu schmalen. (cite mobileFrontier)

% subsubsection entferne_das_fett (end)

\subparagraph{Schnelles Zugriff auf wichtige informationen} % (fold)
\label{subp:subparagraph_name}

Benutzer will nicht immer in das innenleben von Programmen eintauchen, nur um kleine wichtige bruchteile der information zu gewinnen. Deshalb sollte man wichtige Informationen schon etwa beim einem Streifblick erkenbar sein(mobileFrontier)\cite{Neil:2012uf}. Als Beispiel dient hier der Springboard Muster, wie etwa bei iPhone Homescreen, in bei der Piktogram von Email die Anzahl der neuen Emails sehen kann oder beim Kalender App, wo man die Datum sowie Anzahl der Terminanforderung sehen kann (screenShots). Auch die Facebook App bedient sich an diesen muster (screenShot). Für Informationen, die Leistungen oder Ergebniss darstellen, kann man mithilfe eines Dashboard Musters erste Einstieg für Navigation gewähren. Dieses Muster ist sehr hilfreich für Analytische Applikationen.

\subparagraph{Reduziere das Wühlen} % (fold)
\label{subp:reduziere_das_w_hlen}
Text über struktur der Programme. Bezug auf enferne das fett.
- Kleine herarchie von funktionaliät, keine lange phade

\subparagraph{Ermögliche eine Fortsetzung} % (fold)
\label{subp:erm_gliche_eine_fortsetzung}

- In mobilen kontext wird man ständig unterbrochen
- Ausfühlen von Formularen, 


% subparagraph reduziere_das_w_hlen (end)

% subparagraph erm_gliche_eine_fortsetzung (end)Ermögliche eine Fortsetzung
\subparagraph{Benutze Zeit als Ordnungsprinzip} % (fold)
\label{subp:benutze_zeit_als_ordnungsprinzip}

% subparagraph benutze_zeit_als_ordnungsprinzip (end)
\subparagraph{Reduziere kognitive Aufgaben } % (fold)
\label{subp:reduziere_kognitive_aufgaben_}

% subparagraph reduziere_kognitive_aufgaben_ (end)Reduziere kognitive Aufgaben 
\subparagraph{Designe für Unterbrechungen} % (fold)
\label{subp:designe_f_r_unterbrechungen}

% subparagraph designe_f_r_unterbrechungen (end)Designe für Unterbrechnungen 
\subparagraph{Fokussiere auf Erfahrungen, die nur mobil auftreten können} % (fold)
\label{subp:fokussiere_auf_erfahrungen_die_nur_mobil_auftreten_k_nnen}

% subparagraph fokussiere_auf_erfahrungen_die_nur_mobil_auftreten_k_nnen (end)

% subparagraph subparagraph_name (end)

\subsection{Gestaltung von Diensten} % (fold)
\label{sub:gestaltung_von_diensten}

- Hier ab seite 90 (mobileFrontier) paar stichpunkte nehmen. Ist aber mehr von presi http://www.slideshare.net/preciousforever/patterns-for-multiscreen-strategies abgeleitet. 

Themen: Wechsel zwichen Geräten, da man immer im bewegung ist. So sollen Dienste gerätewechsel unterstützen. Auch die aufgaben, die man erledigt, sollten auf anderen gerät oder ort weiterzzführbar sein.

Dienste sollen auch so angepasst werden, dass die im Kernaufgaben auf beliebigen gerät laufen können.

Man sollte auch dabei aufpassen, dass man mit eine Symbiose von geräten vorhanden sein sollte. Wie etwa dass man mit einen iPhone einen Fernseher steuern kann. Also anhand des Kontextes und Ortes dienste angeboten werden sollten

\subparagraph{Dienstkoherenz und Synchronisation} % (fold)
\label{subp:diensmobilit_t}

Beispiel: Evernote

- Erreichung von diesnten von beliebigen gerät.

% subsection dienstmobilit_t (end)
\subparagraph{Gerätemobilität} % (fold)
\label{subp:ger_temobilit_t}

- Gerätemobilität: Benutzung von diensten in Jeden gerät. 

% subparagraph ger_temobilit_t (end)

\subsection{Wearable Computers} % (fold)
\label{sub:wearable_computers}

- Kurze einführung, bedienung. HUD und Google Glasses

- Wichtige sachen, wo man aufpassen sollte bei der Entwicklung von GUIs für kleine bildshirme

- Probleme beim Wearable COmputers, etwa attention fragmentation

% subsection wearable_computers (end)

% subsection gestaltung_von_diensten (end)